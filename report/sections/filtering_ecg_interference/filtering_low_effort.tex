\subsection{Adaptive Filtering of Low-Effort EMG}

\subsubsection{Reference Signal Generation and Alignment}
To apply the LMS adaptive filter, a reference signal $x(n)$ correlated with the interference is required. 
We generated a train of unit impulses by cross-correlating the \texttt{avg\_beat} template with the 
filtered EMG signal to detect all heartbeats.

A critical design consideration was the \textbf{alignment} of these impulses. Since the adaptive filter 
is causal, the impulse must trigger the filter at the \textit{start} of the beat window, not at the peak. 
Therefore, impulses were placed approximately 200 samples before the QRS peak. Figure \ref{fig:alignment} 
confirms this alignment, showing the impulses (red triangles) start before the QRS complexes.

\begin{figure}[H]
    \centering
    \includegraphics[width=1.0\textwidth]{img/impulse_alignment_check.png}
    \caption{The reference impulses (red) start at the onset of the cardiac cycle, enabling causal reconstruction of the artifact.}
    \label{fig:alignment}
\end{figure}

\subsubsection{LMS Filtering Results}
The LMS algorithm was applied with a filter order of $N=600$. We evaluated three step sizes: $\mu \in \{0.001, 0.01, 0.1\}$.

\begin{figure}[H]
    \centering
    \includegraphics[width=1.0\textwidth]{img/lms_low_effort_results.png}
    \caption{Filtered Low-Effort EMG signals. $\mu=0.1$ (bottom) removes the spikes and keeps the signal with good resolution.}
    \label{fig:low_effort_res}
\end{figure}

\begin{figure}[H]
    \centering
    \includegraphics[width=1.0\textwidth]{img/lms_low_effort_artifacts.png}
    \caption{Estimated artifacts $y(n)$. It can be observed that the best results are achieved with the parameter $\mu=0.1$ (below).}
    \label{fig:low_effort_art}
\end{figure}

The results indicate that $\mu = 0.1$ provides the better results. Smaller step size ($\mu=0.001$) results in slower 
convergence, failing to estimate the full amplitude of the first few beats.