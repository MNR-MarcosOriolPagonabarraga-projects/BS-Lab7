
\begin{center}
    \LARGE \textbf{ADAPTIVE FILTERING WITH LMS ALGORITHM}
\end{center}

\section{Introduction}

The accurate recording and analysis of bioelectric signals, 
such as the Electromyogram (EMG), are \textbf{fundamental to 
physiological monitoring and clinical diagnostics}. 
However, these signals are frequently contaminated by 
significant non-biological interference, making accurate 
analysis challenging. Effective filtering 
techniques are therefore \textbf{essential to isolate the signal} 
of interest from unwanted noise.

One of the most pervasive sources of interference in EMG 
recordings is the \textbf{power line artifact}, 
typically oscillating at 50 Hz or 60 Hz. Since the 
power line frequency falls within the operational bandwidth of 
the EMG signal, its removal is critical, yet challenging, 
as traditional \textbf{linear time-invariant (LTI) filters} 
can inadvertently remove desired biological signal components 
alongside the noise.

This document explores the application of \textbf{Adaptive 
Filtering} using the \textbf{Least Mean Squares (LMS) algorithm} 
 as a dynamic solution to interference cancellation. 
Unlike fixed-coefficient LTI filters (such as the Butterworth 
Notch filter), the LMS algorithm adjusts its 
filter weights iteratively to track and minimize the 
interference component in the primary signal.

The work is structured in two main parts:
\begin{enumerate}
    \item \textbf{Filtering Power Line Interference}: 
    We apply the LMS algorithm to remove 60 Hz power line 
    noise from EMG signals. The performance is 
    evaluated using the \textbf{Energy Spectral Density (ESD)} 
    and compared against a fixed-coefficient LTI 
    filter to assess the trade-offs in noise 
    reduction and signal preservation.
    \item \textbf{Filtering ECG Interference in EMG Signals} : 
    We address the more complex issue of 
    removing cardiac activity (ECG) artifacts from EMG 
    signals. Since a separate ECG reference 
    signal is not available, we synthesize the required 
    reference by extracting an \textbf{"average ECG beat"} 
    and using cross-correlation to generate 
    an impulse train.
\end{enumerate}

Ultimately, this study aims to analyze the \textbf{convenience} 
and performance of adaptive filtering in handling both simple 
(sinusoidal, 60 Hz) and complex (morphological, 
ECG) interference components in biomedical signals.
