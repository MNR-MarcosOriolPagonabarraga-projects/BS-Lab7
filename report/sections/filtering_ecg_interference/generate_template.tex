\subsection{Generation of Average ECG Beat Template}
Using the manually selected indices, we extracted segments of the raw EMG signal to construct an average beat. 
A window of 600 samples was defined around each peak: 199 samples before the peak and 400 samples after. 
This window length (600 ms) is sufficient to capture the P-wave, QRS complex, and T-wave.

\begin{lstlisting}[language=Matlab, caption={Extraction of the average ECG beat}, label={lst:avg_beat}]
function avg_beat = extract_average_beat(signal, peak_locs, window_pre, window_post)

    num_peaks = length(peak_locs);
    beat_segments = [];
    
    for i = 1:num_peaks
        start_idx = peak_locs(i) - window_pre;
        end_idx = peak_locs(i) + window_post;
        
        if start_idx > 0 && end_idx <= length(signal)
            segment = signal(start_idx:end_idx);
            beat_segments = [beat_segments; segment(:)];
        end
    end
    
    avg_beat = mean(beat_segments, 1);
end
\end{lstlisting}

The resulting template, shown in Figure \ref{fig:avg_beat}, displays a distinct cardiac morphology with 
minimal high-frequency noise, validating the manual selection strategy.

\begin{figure}[H]
    \centering
    \includegraphics[width=0.8\textwidth]{img/average_ecg_beat.png}
    \caption{Average ECG beat template derived from the 15 manually selected indices. The P, QRS, and T 
    waves are clearly visible.}
    \label{fig:avg_beat}
\end{figure}