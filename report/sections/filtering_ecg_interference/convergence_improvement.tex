\subsection{Convergence Improvement}

\subsubsection{The Initialization Problem}
Standard adaptive filters (LMS) are typically initialized with a weight vector of zeros ($w[0] = 0$). This assumes 
no prior knowledge of the interference signal. Theoretically, the filter requires a "learning period" (convergence time) 
to adjust its weights, during which the interference is not removed.

\subsubsection{Smart Initialization with Template}
To eliminate this potential delay, we proposed a "Smart Initialization" strategy. Using the computed \texttt{avg\_beat} 
template (which represents the optimal impulse response), we initialized the filter weights: 
$$ w_{initial} = \text{avg\_beat} $$
By seeding the LMS algorithm with this prior knowledge, the filter starts in a state closer to the optimal solution.

\subsubsection{Results Analysis}
Figure \ref{fig:conv} compares the filtered output of both methods during the first second of recording, using a step 
size of $\mu=0.1$.

\begin{itemize}
    \item \textbf{Observation:} Contrary to the theoretical expectation of a distinct "learning curve," both the Null 
    and Smart initializations produce very similar results. The first QRS complex is attenuated in both cases, with no 
    significant visual difference in the residual artifact.
    
    \item \textbf{Explanation of Fast Convergence:} The chosen step size ($\mu=0.1$) is sufficiently large to allow the 
    LMS algorithm to adapt extremely rapidly. In the Null case, the filter weights adjust from zero to the required magnitude 
    within the duration of the first QRS complex itself (approx. 50-100 ms). This ultra-fast adaptation renders the 
    initialization phase negligible for visual inspection.
    
    \item \textbf{Template Mismatch:} Furthermore, the Smart Initialization relies on an \textit{average} beat template. 
    Since the specific amplitude and shape of the very first heartbeat in the raw signal naturally deviate from this average, 
    the "Smart" filter still produces a residual error (due to imperfect subtraction). This residual is comparable in magnitude 
    to the initial error of the rapidly adapting Null filter, leading to the observed similarity.
\end{itemize}

\begin{figure}[H]
    \centering
    \includegraphics[width=0.8\textwidth]{img/convergence_comparison_filtered.png}
    \caption{Comparison of filter convergence during the first second. \textbf{Top:} Null Initialization. \textbf{Bottom:} 
    Smart Initialization. Due to the high adaptation speed ($\mu=0.1$) and natural amplitude variance of the first beat, the 
    visual performance is nearly identical in both cases, with no clear advantage observed for the template-based initialization 
    in this specific scenario.}
    \label{fig:conv}
\end{figure}