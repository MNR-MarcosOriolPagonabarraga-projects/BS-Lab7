\subsection{Signal Visualization and Manual Beat Selection}
The dataset \texttt{adapecg.mat} was loaded, containing electromyography (EMG) signals recorded from 
the sternomastoid muscle during five respiratory cycles at two different inspiratory effort levels: 
low and high.

As shown in Figure \ref{fig:raw_emg}, the raw EMG signals exhibit significant cardiac interference. 
This is characterized by sharp, periodic QRS complexes contaminating the physiological signal. In the 
"High Effort" scenario (bottom plot), the EMG amplitude increases significantly during inspiratory bursts,
 partially masking the ECG artifacts.

To facilitate the extraction of a clean cardiac template, the EMG signal (Low Effort) was first filtered 
using a $6^{th}$ order Butterworth low-pass filter with a cutoff frequency of 70 Hz. 
Following the laboratory guidelines, 15 QRS peaks were manually selected during "quiet" expiratory phases 
to avoid high-amplitude EMG noise associated with inspiration. The manually identified indices 
were: $1823, 2459, \dots, 17035$.

\begin{figure}[H]
    \centering
    \includegraphics[width=1.0\textwidth]{img/raw_emg_signals.png}
    \caption{Raw EMG signals recorded during Low Effort (top) and High Effort (bottom). Note the periodic ECG artifacts visible throughout the recordings.}
    \label{fig:raw_emg}
\end{figure}